% Created 2021-01-27 Mi 21:01
% Intended LaTeX compiler: pdflatex
\documentclass[11pt]{article}
\usepackage[utf8]{inputenc}
\usepackage[T1]{fontenc}
\usepackage{graphicx}
\usepackage{grffile}
\usepackage{longtable}
\usepackage{wrapfig}
\usepackage{rotating}
\usepackage[normalem]{ulem}
\usepackage{amsmath}
\usepackage{textcomp}
\usepackage{amssymb}
\usepackage{capt-of}
\usepackage{hyperref}
\usepackage{minted}
\usepackage{geometry}\geometry{a4paper,left=15mm,right=20mm,top=20mm,bottom=30mm}
\newcommand{\R}{\mathbb{R}} \newcommand{\C}{\mathbb{C}}
\usepackage{extarrows} \usepackage{mathtools} \usepackage[utf8]{inputenc}\usepackage[T2A]{fontenc}
\renewcommand{\phi}{\varphi} \newcommand{\parti}[2]{\frac{\partial #1}{\partial #2}}
\date{\today}
\title{}
\hypersetup{
 pdfauthor={},
 pdftitle={},
 pdfkeywords={},
 pdfsubject={},
 pdfcreator={Emacs 27.1 (Org mode 9.3)}, 
 pdflang={English}}
\begin{document}



\section*{might be useful}
\label{sec:org2cda3a3}
\[ \frac{d}{dx} \left (\int_{0}^{x} f(x,y)\,dy \right) = f\big(x,x) + \int_{0}^{x}\frac{\partial}{\partial x} f(x,y) \,dy\]

\medskip

care e obtinuta din formula Leibniz:
\[ \frac{d}{dx} \left (\int_{a(x)}^{b(x)}f(x,t)\,dt \right) = f\big(x,b(x)\big)\cdot \frac{d}{dx} b(x) - f\big(x,a(x)\big)\cdot \frac{d}{dx} a(x) + \int_{a(x)}^{b(x)}\frac{\partial}{\partial x} f(x,t) \,dt\]
\begin{itemize}
\item laplace operator (laplacian): \(\Delta = \nabla^2 = \nabla \cdot \nabla = \displaystyle \sum_i \dfrac{\partial^2}{\partial x_i^2}\)
\item field therory shit: (curl=rotor)\\
\(\operatorname{div}  \, \operatorname{grad} f          \equiv \nabla \cdot  \nabla f \equiv \nabla^2 f\)\\
\(\operatorname{curl} \, \operatorname{grad} f          \equiv \nabla \times \nabla f = \mathbf 0\)\\
\(\operatorname{div}  \, \operatorname{curl} \mathbf{A} \equiv \nabla \cdot  (\nabla \times \mathbf{A}) = 0\)\\
\(\operatorname{curl} \, \operatorname{curl} \mathbf{A} \equiv \nabla \times (\nabla \times \mathbf{A}) = \nabla (\nabla \cdot \mathbf{A}) - \nabla^2 \mathbf{A}\)\\
\(\nabla^2 (f g) = f \nabla^2 g + 2 \nabla f \cdot \nabla g + g \nabla^2 f\)
\item Gauß-Остроградский:
\[ \int_\Omega \parti{u}{x_i} = \int_{\partial \Omega}u \cdot \nu_i d \sigma \]
also:
\end{itemize}
\[ \int_\Omega \operatorname{div} F (x) dx = \int_{\partial \Omega} F  \cdot \nu d \sigma \]
\begin{itemize}
\item Green formula:
\end{itemize}
\[\int_\Omega (\nabla u \cdot \nabla v  + v \Delta u) dx = \int_{\partial \Omega} v \sum_i \parti{u}{x_i} \nu_i d \sigma \]
\begin{itemize}
\item Green formula 1:
\end{itemize}
\[ -\int_\Omega \Delta u v dx = \int_\Omega \nabla u \cdot \nabla v dx - \int_{\partial \Omega} \parti{u}{\nu} v  d \sigma \]
\begin{itemize}
\item Green formula 2:
\end{itemize}
\[ \int_\Omega (\Delta u v u\Delta v) dx = \int_{\partial \Omega} \parti{u}{\nu} v - u \parti{v}{\nu}  d \sigma \]
\begin{itemize}
\item convoluție:
\[ f*g (x) = \int_{\R^d} f(x-y) g(y)dy \]
\item REMEMBER THE NORM FOR THE FOURIER THING:
\[ u(x) = \sum_k \frac{1}{\|\phi_k\|^2} \int_a^b u(x) \phi_k(x) dx \]
\item fourier desmos: \url{https://www.desmos.com/calculator/xkc2e0emnm}

\item Ec dif ordin I
\end{itemize}
\[x' = a(t)x + b(t) \quad \quad x = e^{\int a(t)dt} \left( C + \int e^{-\int a(t) dt} b(t) dt \right)\]
\subsection*{coś, sin stuff:}
\label{sec:orgb80c288}
\begin{itemize}
\item \(e^{ix} = \cos x + i \sin x\)
\item \(\sin (k \pi) = 0\)
\item \(\cos (k \pi) = (-1)^k\)
\item \(\int \sin = - \cos\)
\item \(\int \cos = \sin\)
\item \(\sin' = \cos\)
\item \(\cos' = - \sin\)
\item \(\sin x = \dfrac{e^{ix}-e^{-ix}}{2i}\)
\item \(\cos x = \dfrac{e^{ix}+e^{-ix}}{2}\)
\end{itemize}
\subsubsection*{product to sum}
\label{sec:org44079ad}
\begin{itemize}
\item \(2\cos \theta \cos \varphi = {{\cos(\theta - \varphi) + \cos(\theta + \varphi)}}\)
\item \(2\sin \theta \sin \varphi = {{\cos(\theta - \varphi) - \cos(\theta + \varphi)} }\)
\item \(2\sin \theta \cos \varphi = {{\sin(\theta + \varphi) + \sin(\theta - \varphi)} }\)
\item \(2\cos \theta \sin \varphi = {{\sin(\theta + \varphi) - \sin(\theta - \varphi)} }\)
\item \(\cos^2 \varphi = \frac{1+\cos(2\varphi)}{2}\)
\item \(\sin^2 \varphi = \frac{1-\cos(2\varphi)}{2}\)
\end{itemize}

\subsubsection*{sum to product}
\label{sec:orgc83b924}
\begin{itemize}
\item \(\sin \theta \pm \sin \varphi = 2 \sin\left( \dfrac{\theta \pm \varphi}{2} \right) \cos\left( \dfrac{\theta \mp \varphi}{2} \right)\)
\item \(\cos \theta + \cos \varphi = 2 \cos\left( \dfrac{\theta + \varphi} {2} \right) \cos\left( \dfrac{\theta - \varphi}{2} \right)\)
\item \(\cos \theta - \cos \varphi = -2\sin\left( \dfrac{\theta + \varphi}{2}\right) \sin\left(\dfrac{\theta - \varphi}{2}\right)\)
\end{itemize}


\section*{conditions}
\label{sec:org2f5f0d7}
\begin{itemize}
\item dirichlet: \(u(x) = f(x)\) pe \(\partial \Omega\)
\item newman: \(\parti{u}{\nu}(x) = f(x)\) pe \(\partial \Omega\)
\item robin: \(\parti{u}{\nu}(x) +u(x) = f(x)\) pe \(\partial \Omega\)
\end{itemize}
\section*{misc}
\label{sec:orgceab0b8}
\begin{itemize}
\item proiectie pe subspatii inchise:
\end{itemize}
\[ \exists ! Pu \in V  \text{ aî } \| P u - u \|  = \inf_{v \in V} \| v - u \| \]
\begin{itemize}
\item în plus \((u-Pu)\perp V\) (ie \(\langle u-Pu, v \rangle = 0,~ \forall v \in V\))
\end{itemize}
\begin{itemize}
\item bessel inequality: \(V = \operatorname{span} \{ v_1,\ldots, v_n \}\)
\[ \| u\|^2 \geq \|P_V u\|^2 = \sum_{j=1}^n \frac{|\langle u, v_j\rangle|^2}{\|v_j\|^2} \]
\item daca are loc ineg parseval și \(\{f_j\}\) ortongolală, at e bază Hilbertiană
\item scalar product with functions \(f: [a,b] \to \mathbb{C} \in L^2([a, b])\):
\item weak convergence:
\[ \langle f, g\rangle = \int_a^b f(x) \overline{g(x)} dx  \]
\[ u^n \rightharpoonup u \text{ dacă} \langle u^n-u, v\rangle \to 0, \forall v \in H \]
\end{itemize}

\section*{Sturm-Liouville - S1}
\label{sec:org2b4f4eb}
\subsection*{valori proprii}
\label{sec:org9db0e4b}
\begin{itemize}
\item c3 - pg 5
\item Melnig thing pg 7

\item do the \(|\cdot \varphi, \int\) to get \(\lambda \geq 0\)
\item if \(\varphi = 0\) we ignore that one. we dont want null solutions
\end{itemize}

\begin{itemize}
 \item we get the characteristic 
 equation\footnote{\url{https://en.wikipedia.org/wiki/Characteristic_equation_(calculus)}} (polinom caracteristic:
  \[ \varphi''(x) + \lambda \varphi(x) =0  \text{ becomes } r^2 + \lambda\cdot 1 = 0 \]  
\end{itemize}
\begin{itemize}
\item and we get the functions of form (may differ depending on the characteristic equation):
\[\left \{ e^{r_i}, \ldots, x^m e^{r_i} \right\} \] 
or, for our example:
\[ \sin(\sqrt{\lambda} x), \cos(\sqrt{\lambda} x) \] 
so 
\[\varphi(x) = \alpha \sin(\sqrt{\lambda} x) + \beta \cos(\sqrt{\lambda} x) \]
\item with the initial conditions: \(\varphi(0) = \varphi(l) = 0\)
we get some restrictions for \(\alpha\) and \(\beta\)
\end{itemize}
and, tada, ya get some \(\lambda_k, \varphi_k\)

\section*{Green's function:}
\label{sec:org422db96}
\begin{itemize}
\item for n-th order differential equations:
see green-kurzgesagt
\end{itemize}
\section*{separation of variables}
\label{sec:orgea0b630}
\begin{itemize}
\item see s6 - pg 2
\item we have:
\end{itemize}
\[
\begin{cases}
-\Delta u = f,\quad \text{în }\Omega = (a, b) \times (c, d)\\
\text{some condition like } u = 0,\quad \text{pe }\partial\Omega
\end{cases}
\]
\begin{itemize}
\item we write stuff with respect to \(x\):
\end{itemize}
\[
\begin{cases}
- \phi'' = \lambda \phi,\quad \text{în }\Omega = (a, b)\\
\text{some condition like } u(a) = u(b) = 0
\end{cases}
\]
\begin{itemize}
\item and we get some eigen functions and values: 
\(\{\phi_k\}\), \(\{\lambda_k\}\)
\item we write things with the new functions:
\[ u(x, y) = \sum^\infty_k u_k(y) \phi_k(x) \]
\[ u_{xx} = ..., u_{yy} = ...\]
\[ f(x, y) = \sum f_k(y) \phi_k(x) = \sum \frac{1}{\| \phi_k\|^2} \left( \int_a^b f(t, y) \phi_k(t) dt \right) \phi_k(x)\]
\end{itemize}
\begin{itemize}
\item then we solve it for some \(k\)
\[
\begin{cases}
-\Delta u_k(y) = f_k(y),\quad \text{în }(c, d)\\
\text{some condition like } u(c) = u(d) = 0
\end{cases}
\]
and we get
\[ u_k(y) = \int_c^d G_k(y, s) f_k(s) ds \]
\item sum things together and we get a \(G \big((x, y), (t, s)\big)\):
 \[ u(x, y) = \int_c^d \sum_k G_k(y, s) \frac{1}{\| \phi_k\|^2} \left( \int_a^b f(t, y) \phi_k(t) \, dt \right) \phi_k(x)\, ds \]
aka
\end{itemize}
\[ u(x, y) = \int_c^d \int_a^b \left(\sum_k G_k(y, s) \frac{1}{\| \phi_k\|^2}  \phi_k(t) \phi_k(x) \right) f(t, s) \, dt \, ds \]
  and, tada
  \[G \big((x, y), (t, s)\big) = \sum_{k=1}^\infty G_k(y, s) \frac{1}{\| \phi_k\|^2}  \phi_k(t) \phi_k(x) \]

\subsection*{eigen values for op laplace}
\label{sec:org37d1342}
\begin{itemize}
\item s7 pg 6
\item tl;dr we split it in 2 and get sum the eigenvalues
\end{itemize}

\section*{max principle and stuff}
\label{sec:org52461f7}
\begin{itemize}
\item \(\Delta\) = op laplace
\item \(\Delta u = 0\) means \(u\) armonică
\item \(\Delta u \geq 0\) means \(u\) subarmonică
\item \(\Delta u \leq 0\) means \(u\) super-armonică
\end{itemize}
\subsection*{The actual thing}
\label{sec:org5d06f89}
\begin{itemize}
\item s9 pg 3
\end{itemize}
Dacă \(C^2(\Omega) \cap C(\bar{\Omega})\) și \(\Delta \geq 0 \text{ în } \Omega\)  at:\\
\(\sup\limits_{\bar{\Omega}} u = \sup\limits_{\partial \Omega} u\)\\
și dacă \(\exists \bar{x} \in \Omega\) aî \(u(\bar{x}) = \sup\limits_{\bar{\Omega}} u\) at \(u \equiv\) const
\subsection*{unicitatea sol dirichlet}
\label{sec:orgcc3b1c1}
\begin{itemize}
\item s9 pg 4
\item übermelnig 109
\item tl;dr if we have
\[ 
\begin{cases}
\Delta u = f, & \text{ în } \Omega \subseteq \R^d\\
u = f, & \text{ pe } \partial \Omega
\end{cases}
\]
we give \(v = u_1 - u_2\)
\end{itemize}
\[ 
\begin{cases}
   \Delta v = 0, & \text{ în } \Omega \subseteq \R^d\\
   v = 0, & \text{ pe } \partial \Omega
   \end{cases}
   \]
and by the "max principle" we have: \(\sup\limits_\Omega v \leq 0\)
we switch \(u_1\) and \(u_2\) and we get \(v = 0\) ie \(u_1 = u_2\)

\subsection*{strong max principle (aka pp Hopf)}
\label{sec:org1dd6075}
Dacă \(\bar{x} \in \partial \Omega\) și \(u(\bar{x}) = M\) at:\\
\(\displaystyle \parti{u}{\nu} (\bar{x}) >0\)\\
sau \(\displaystyle \parti{u}{\nu} (\bar{x}) = 0\) și \(u \equiv M\) în \(\Omega\)   
\section*{variational principle}
\label{sec:org2fed501}
\begin{itemize}
\item Fundamental sol for laplace:
\[
E(x) = \begin{dcases}
\frac{1}{2\pi} \ln|x|, & d=2\\
-\frac{1}{(d-2)\omega_d|x|^{d-2}},& d> 2
\end{dcases}
\]
unde (aka aria bilei unitate):
\[\omega_d = \mu_{d-1}(\partial B_1) = \int_{\partial B_1}  1 d \sigma \]
\end{itemize}
\begin{itemize}
\item btw: \(E(x) = E(|x|)\)
\item Riemann-green: (c8)
\[
\int_\Omega E(x-y) \Delta u(y) dy - \int_\Omega E(x-y) \parti{u}{\nu_y}(y) d\sigma_y +
\int_{\partial\Omega}  \parti{}{\nu_y}E(x-y) u(y) d\sigma_y =
\begin{cases}
u(x), & x \in \Omega,\\
\frac{1}{2} u(x), & x \in \partial\Omega,\\
0, & x \in \R^d \setminus \bar{\Omega}
\end{cases}
\]
\end{itemize}
\subsection*{actual solving - übermelnig - pg 115,117,118, 121,122:}
\label{sec:org939fa45}
\begin{itemize}
\item Sol variationala e sol clasica
\end{itemize}
având:
 \[
\begin{cases}
 - \delta u = f, &\Omega,\\
 u= g_1, & \Gamma_1,\\
 \parti{u}{\nu}= g_2, & \Gamma_2,\\
 \parti{u}{\nu}+u= g_3, & \Gamma_3\\
\end{cases}
 \]
definim
\[V = \left\{v \in C^1_p(\Omega) \mid v=0 \text{ pe } \partial \Omega \right\}\]
calcul formal (via Green formula 1; it's exacly this), \(u\in C^2(\Omega)\):
\[-\int_\Omega \Delta u v\, d\mu = \int_\Omega \nabla u\nabla v\,d\mu-
 \int_{\partial\Omega} \parti{u}{\nu} v\,d\sigma\]

 then we write:
 \[
 \int_{\partial\Omega} \parti{u}{\nu} v\,d\sigma =
 \int_{\Gamma_1} \parti{u}{\nu} v\,d\sigma + 
 \int_{\Gamma_2} \parti{u}{\nu} v\,d\sigma + 
 \int_{\Gamma_3} \parti{u}{\nu} v\,d\sigma
\]
\[
 \int_{\partial\Omega} \parti{u}{\nu} v\,d\sigma =
 \int_{\Gamma_1} \parti{u}{\nu} v\,d\sigma + 
 \int_{\Gamma_2} g_1 v\,d\sigma + 
 \int_{\Gamma_3} (g_3-u) v\,d\sigma
\]
split it into\\
\(a(u, v)\) -simetrica, biliniara, and \(\ell (v)\), liniara, cont

for unicitate \(w=u_1-u_2\), \(v = w\) and we get \(a(w, w) = 0\)

sigh, see the pages mentioned above
\section*{fourier transform:}
\label{sec:org7c17071}
\begin{itemize}
\item def: \[\hat{f}(\lambda) = \frac{1}{2\pi} \int_{-\infty}^{\infty} f(x)\ e^{- i \lambda x}\,dx\]
\item see \url{https://en.wikipedia.org/wiki/Fourier\_transform\#Functional\_relationships,\_one-dimensional}
\item \(\hat{u}^{(k)}(\lambda) = \widehat{[(-ix)^k u(x) ]} (\lambda)\)
\item \(\widehat{u^{(k)}}(\lambda) = (2\pi{}i \lambda)^k \hat{u}(\lambda)\)
\item \(\widehat{u*v}(\lambda) = \hat{u}(\lambda)\hat{v}(\lambda)\)
\item \(\widehat{u\cdot v}(\lambda) = \hat{u}(\lambda)*\hat{v}(\lambda)\)
\item \(\widehat{u_x}(\lambda) = i \lambda \hat{u}(\lambda)\)
\item \(\widehat{u_{xx}}(\lambda) = - \lambda^2 \hat{u}(\lambda)\)
\item \(\widehat{u_t}(\lambda) = \parti{}{t} \hat{u}(\lambda)\)
\item \(\widehat{u_{tt}}(\lambda) = \parti{^2}{t^2} \hat{u}(\lambda)\)
\item \(\widehat{u(x-a)}(\lambda) = e^{-ia \lambda} \hat{u} (\lambda)\)
\item \(\widehat{\hat{u}(x)}(\lambda) = \hat{u} (-\lambda)\)
\item \(\widehat{\hat{u}(ax)}(\lambda) = \frac{1}{|a|} \hat{u}\left (\frac{\lambda}{a}\right)\)
\end{itemize}

\section*{toc}
\label{sec:org7163f9e}
\subsection*{course}
\label{sec:org98f85f9}
\begin{itemize}
\item C1: basic shit
\item C2: 
\begin{itemize}
\item basic shit (prod scalar and norm)
\item projections
\item besel inequality
\end{itemize}
\item C3:
\begin{itemize}
\item more besel
\item hilbert basis
\item problem with Green's function
\item hilbert spaces examples
\end{itemize}
\item C4:
\begin{itemize}
\item proprietati Green's thing - pg 2
\item Riesz  representation theorem - pg 5 (dual stuff)
\item autoadjunct daca \(T = T^*\)
\end{itemize}
\item C5:
\begin{itemize}
\item weak convergence
\item hilbert basis proprierties \& stuff
\end{itemize}
\item C6: 
\begin{itemize}
\item more weird abstract shit
\item sturm liouville in general form - pg 11
\end{itemize}
\item C7:
\begin{itemize}
\item differential subvariety stuff
\item green's formulas
\item convolutions
\item that weird fundam\(E\)ntal thing
\end{itemize}
\item C8:
\begin{itemize}
\item unicitate, existenta, repr integrala, dependenta de datele pb, approx numerica
\item fundamental solution for \(\Delta\) - op laplace
\item riemann-green
\end{itemize}
\item C9:
\begin{itemize}
\item riemann green again
\item unicitate, existenta, repr integrala, dependenta de datele pb, approx numerica, but actually done
\end{itemize}
\item c10
\begin{itemize}
\item pp maxim general
\item sol variationale, finally pg 8
\end{itemize}
\item s11
\end{itemize}
\subsection*{seminaries}
\label{sec:org7bbca87}
\subsubsection*{S1}
\label{sec:org247a760}
\begin{itemize}
\item tl;dr normal differential equations
\end{itemize}
\[
\begin{cases}
u'_k(t) + \lambda_k u_k(t) = f_k(t), t>0\\
u_k(0) = u_k^0
\end{cases}
\]\[
u_k(t) = e^{-\lambda_kt} u_k^0 + \int_0^t \exp(-\lambda_k(t-s)) f_k(s)ds
\]
\begin{itemize}
\item sturm-liouville stuff
\end{itemize}
\subsubsection*{S2, s3}
\label{sec:org38ec610}
\begin{itemize}
\item sturm-liouville and fourier exercises
\end{itemize}
\subsubsection*{s4:}
\label{sec:orga7e986a}
\begin{itemize}
\item met sep variabilelor pg 4
\item fundamental solution pg 10
\end{itemize}
\subsubsection*{s5}
\label{sec:org0972666}
\begin{itemize}
\item green shit
\end{itemize}
\subsubsection*{s6, s7}
\label{sec:org725da9f}
\begin{itemize}
\item separation of variabiles for sturm-liouvile problems + green - pg 3
\end{itemize}
\subsubsection*{s7}
\label{sec:org2de8fbf}
\begin{itemize}
\item solving eigen-value problems for \(\Delta\)
\end{itemize}
\subsubsection*{s8}
\label{sec:orgfd2fd63}
\begin{itemize}
\item recapitulare
\end{itemize}
\subsubsection*{s9, s10, s11}
\label{sec:org55e095d}
\begin{itemize}
\item pp de maxim +aplicatii
\end{itemize}
\subsubsection*{s11}
\label{sec:orgd787a1f}
\begin{itemize}
\item that weird fundam\(E\)ntal thing pg 11
\end{itemize}
\subsubsection*{s12}
\label{sec:orgb774d2d}
\begin{itemize}
\item variational thing
\end{itemize}
\subsection*{that old book}
\label{sec:org0d61f66}
\begin{itemize}
\item green - pg 39
\end{itemize}
\subsection*{melnig thing}
\label{sec:org3562d2e}
\begin{itemize}
\item 7 - val proprii
\item 15 - parseval stuff
\end{itemize}

\subsection*{über-melnig thing - maed bai benni}
\label{sec:org477c09e}
Most of the stuff are seen in the melnig seminaries:

Par example
\subsubsection*{Ex 1: Sturm- Liouville: page 6 - 14}
\label{sec:org5c3efe5}
Replace a with smth else ofc.

\smallskip
\subsubsection*{Ex2: Ar ca \ldots{} ortogonale   page 4 -  6}
\label{sec:org799a12c}



Also Id Parseval + Dezv in serii Fourier: page 15 - 30
\smallskip

\subsubsection*{Ex3: Metoda separarii variabilelor: mostly from page 42 to -  102}
\label{sec:orgd84226f}

Furthermore, there is the list on which to calculate\ldots{}

Most seen stuff: metoda separarii, pb parabolica: page 32


An example : page 36


Pb hiperbolica: page 47 and 86

Also check Sem9, page 80

\subsubsection*{Ex: 4 problema eliptica la limita: page 91,}
\label{sec:org2cecd06}

Principiul de maxim: page 103

Formularea variationala pt elipsa: page 115, also s13-14 first pages

\subsubsection*{ex 5: page 115}
\label{sec:orgc150b55}
\subsubsection*{Ex 6: transformata fourier: check s14, page 13}
\label{sec:orgd22a09f}

\section*{things to know}
\label{sec:org76267a3}
\begin{itemize}
\item sp Hilbert, serii Fourier, pb Sturm-Liouville
\item separarea variabilelor (pb val proprii, hip, parab, eliptice - serii fourier
\item fct Green (op laplace+ sturm liouville)
\item pp maxim (op eliptici + aplicatii - unicitatea sol si estimari)
\item formularea variationala a pb eliptice (si parab si hip) => sep variabilelor
\item transformata fourier - calcul + cateva proprietati

\item oral: he asks bout some theory bit
\end{itemize}
\end{document}
