% Created 2021-01-17 So 23:28
% Intended LaTeX compiler: pdflatex
\documentclass[11pt]{article}
\usepackage[utf8]{inputenc}
\usepackage[T1]{fontenc}
\usepackage{graphicx}
\usepackage{grffile}
\usepackage{longtable}
\usepackage{wrapfig}
\usepackage{rotating}
\usepackage[normalem]{ulem}
\usepackage{amsmath}
\usepackage{textcomp}
\usepackage{amssymb}
\usepackage{capt-of}
\usepackage{hyperref}
\usepackage{minted}
\usepackage{geometry}\geometry{a4paper,left=15mm,right=20mm,top=20mm,bottom=30mm}
\date{\today}
\title{}
\hypersetup{
 pdfauthor={},
 pdftitle={},
 pdfkeywords={},
 pdfsubject={},
 pdfcreator={Emacs 27.1 (Org mode 9.3)}, 
 pdflang={English}}
\begin{document}


\section*{ex 1 - cele mai mici patrate}
\label{sec:org0632c50}

avem \(x_i\) si \(f(x_i)\) din tabel
\medskip

si avem  \(y_{a, b, c} (x)\) forma functiei pe care vrem sa o aproximam (de ex \(ax^2+bx+c\))
\medskip

definim functia: 
\[ S(x, a, b, c) = \sum_i (f(x_i) - y_{a, b,c} (x_i))^2 \]

si derivam in raport cu variabilele:\\
(daca avem de ex \(a/x+b\) atunci derivam doar in raport cu \(a\) și \(b\)

\medskip

si obtinem un sistem
\[
\begin{cases}
\dfrac{\partial S}{\partial a} = 0\\[1em]
\dfrac{\partial S}{\partial b} = 0\\[1em]
\dfrac{\partial S}{\partial c} = 0
\end{cases}
\]

și tada: solutia sistemului da coeficientii functiei pe care o cauți

\section*{ex 2 - aproximare in medie patratica:}
\label{sec:orgd4fd3b3}
se cere "aproximarea in medie patratica prin polinoame \(Q\)"\footnote{i know, wrong quotes} de grad cel mult \(n\) pe 
\((a, b)\)\footnote{a, b pot fi infinit sau poate sa fie interval inchis} a functiei:

\begin{itemize}
\item cauti "Q polynomials" pe motorul de cautare preferat si gasesti ceva de genul asta:
\end{itemize}
\url{https://mathworld.wolfram.com/LegendrePolynomial.html}
\begin{itemize}
\item te uiti pe siteul ala sa vezi care-i ponderea - in engleza e "weight"
\end{itemize}
si se noteaza \(w(x)\), noi ramanem cu notatia profei si scriem \(p(x)\)
\begin{itemize}
\item si gasesti primele \(n\) polinoame: \(Q_i(x)\)
\item calculezi produsul scalar cu functia ta:
\end{itemize}
\[ c_i = (f, P_i) = \int_a^b p(x) f(x) Q_i(x) dx \]
\begin{itemize}
\item si ai aproximarea:
\end{itemize}
\[ f(x) \approx P(x) = \sum_{i=1}^{n} c_i \cdot Q_i(x) \]
cam ca la EDP, dar mai bine nu ne aducem aminte de asta acum

\section*{ex 3 - derivare numerica}
\label{sec:org5d1e300}
calculati \(y^{(d)}(x_0)\) folosind o formula care foloseste toate elementele din multimea 
\[A = \{y(x) | x \in \{x_i, |, i = 1..n \} \}\]
\begin{itemize}
\item spoiler, \(y\) e acelasi ca la ex1
\item definim \(h_i = x_i - x_0\)
\item si avem sistemul
\end{itemize}
 \begin{pmatrix}
 h_1^0 &\cdots &h_n^0\\
 \vdots& \ddots& \vdots\\
 h_1^n  &\cdots& h_n^n
 \end{pmatrix}
\begin{pmatrix}
 \alpha_1\\
 \vdots\\
 \alpha_n\\
 \end{pmatrix}
 = 
 \begin{pmatrix}
 0\\
 \vdots\\
 d!\\
 \vdots\\
 \alpha_n\\
 \end{pmatrix}

in matricea din dreapta avem \(d!\) pe pozitia \(d+1\) (incepand numaratoarea de la 0)\\
de ex daca avem de calculat \(y'\) atunci \(d=1\) si vine:
\begin{pmatrix}
0\\
1!\\
0\\
\vdots\\
0\\
\end{pmatrix}

\begin{itemize}
\item rezolvam sistemul si avem valori pt \(\alpha_i\)
\item si estimarea e:
\end{itemize}
\[ y^{(d)}(x_0) \approx \sum_{i=1}^n \alpha_i y(x_i) \]
\begin{itemize}
\item restul e:
\end{itemize}
\[R = -\frac{1}{n!} \sum_i \alpha_i \cdot  h_i ^ n \cdot  y^{(n)}(\xi_i)\]
\begin{itemize}
\item restul il putem majora cu:
\end{itemize}
\[|R| \leq \frac{1}{n!} \cdot \max\limits_\xi \left|y^{(n)}(\xi)\right| \cdot \sum_i \alpha_i \cdot  h_i ^ n\]

\section*{ex 4:}
\label{sec:org36e7059}
\begin{itemize}
\item spoiler, \(y\) e acelasi ca la ex1
\end{itemize}
un exemplu:
\url{https://github.com/azbyn/math-made-short/blob/master/calcul\_numeric/gauss\_stuff/gauss\_stuff.pdf}
\end{document}
